\chapter{Einleitung}

Software entwickelnde Firmen stehen vor der stetigen Herausforderung immer komplexere Software in besser Qualit�t, in immer k�rzerer Zeit zu niedrigeren Kosten zu produzieren.
Um diese Herausforderung meistern zu k�nnen, muss eine Vielzahl von Techniken und Methoden eingesetzt werden um den Entwicklungsprozess systematisch, planvoll, effizient und damit letztendlich erfolgreich zu gestalten.

Vor allem die Wiederverwendung bereits existierender Software ist ein wesentliches Mittel um dies zu erreichen. Obwohl die Vorteile von Wiederverwendung leicht einzusehen und nicht von der Hand zu weisen sind, stellt sich die konkrete Wiederverwendung von Software oft problematisch dar. 
Diese H�rden versucht die testgetriebene Wiederverwendung zu nehmen, indem Prinzipien aus der agilen Softwareentwicklung genutzt werden um die Wiederverwendung automatisiert in den Entwicklungsprozess einzubinden. Diese Idee unterscheidet sich erheblich von bereits bekannten M�glichkeiten der Wiederverwendung und ist ein aktuelles Forschungsthema.  

Diese Ausarbeitung hat zum Ziel, die Prinzipien der testgetriebenen Wiederverwendung zu erl�utern sowie den aktuellen Forschungsstand anhand bereits existierender Tools aufzuzeigen.