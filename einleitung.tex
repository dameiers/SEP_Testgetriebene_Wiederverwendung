\chapter{Einleitung}

%- Softwareentwicklung, hart umk�mpft, kurzlebiger markt

%- markt fordert immer qualitativ bessere software, in k�rzerer zeit zu weniger kosten

Software entwickelnde Firmen stehen also vor der oft exestentiellen Herausforderung qualitiv bessere Software in immer k�rzerer Zeit zu niedrigeren Kosten zu produzieren.

Dieser zun�chst paradox erscheinende Sachverhalt f�hrt also zu einer wesentlichen Frage:

Wie kann das erreicht werden?

Eine umfassende Betrachtung dieser Frage w�rde den Rahmen dieser Arbeit sprengen, es ist jedoch offensichtlich das oben genannte Herausforderung nur unter Zuhilfenahme verschiedenster Methodiken zu l�sen ist.

Dazu geh�ren sowohl der Einsatz klar definierter (evtl. sogar agiler) Vorgehensmodelle die Werkzeuge und Methoden zum Requirements-Engineering und/oder der Qualit�tssicherung bereitstellen, als auch Konzepte die den Entwickler bei der t�glichen Arbeit unterst�tzen und somit effizienter werden lassen. Zu letzt genannter Kategorie geh�rt auch die Wiederverwendung bereits existierender Software.

Wiederverwendung von Software ist daher auch jeher ein zentrales Thema der Softwareentwicklung. Mit vielen neuartigen Paradigmen der Softwareentwicklung war stets auch die Hoffnung verbunden die Wiederverwendung von Software zu erleichtern und zu steigern. Objektorientierte Programmierung und service-orientrierte Architekturen sind nur wenige von vielen Beispielen daf�r \autocite[vgl.][]{OS5/2011}    

Ein neuer Ansatz der versucht die Wiederverwendung von Software zu verbessern ist die testgetriebene Wiederverwendung. Hier werden sich Eigenschaften der modernen und agilen Sofwareentwicklung zu nutze gemacht um die Suche nach m�glichen Wiederverwendungskandidaten automatisiert und integriert ablaufen zu lassen. Dieser neuartige Ansatz und deren Umsetzung ist aktueller Forschungsbereich. 

Diese Ausarbeitung hat zum Ziel, den neuartigen Ansatz der testgetriebenen Wiederverwendung und ihre zugrundeliegenden Ideen zu analysieren sowie den aktuellen Forschungsstand anhand bereits entwickelter Tools aufzuzeigen.

Zun�chst werden wir jedoch auf die ben�tigten Grundlagen erl�utern.
  
