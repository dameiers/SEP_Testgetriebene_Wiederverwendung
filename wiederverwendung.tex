\chapter{Wiederverwendung von Software}

%- warum m�ssen wir software wiederverwendent?
	%- Softwareentwicklung, hart umk�mpft, kurzlebiger markt

	%- markt fordert immer qualitativ bessere software, in k�rzerer zeit zu weniger kosten
	


Software entwickelnde Firmen stehen also vor der oft exestentiellen Herausforderung qualitiv bessere Software in immer k�rzerer Zeit zu niedrigeren Kosten zu produzieren.

Dieser zun�chst paradox erscheinende Sachverhalt f�hrt also zu einer wesentlichen Frage:

Kann dies erreicht werden und wenn ja wie?

Eine umfassende Betrachtung dieser Frage w�rde den Rahmen dieser Arbeit sprengen, es ist jedoch offensichtlich das oben genannte Herausforderung nur unter Zuhilfenahme verschiedenster Methodiken zu l�sen ist.

Dazu geh�ren sowohl der Einsatz klar definierter (evtl. sogar agiler) Vorgehensmodelle die unter anderem Werkzeuge und Methoden zum Requirements-Engineering und/oder der Qualit�tssicherung bereitstellen, als auch Konzepte die den Entwickler bei der t�glichen Arbeit unterst�tzen und somit effizienter werden lassen. Zu letzt genannter Kategorie geh�rt auch die Wiederverwendung bereits existierender Software.

Wiederverwendung von Software ist daher auch jeher ein zentrales Thema der Softwareentwicklung. Mit vielen neuartigen Paradigmen der Softwareentwicklung war stets auch die Hoffnung verbunden die Wiederverwendung von Software zu erleichtern und zu steigern. Objektorientierte Programmierung und service-orientrierte Architekturen sind nur wenige von vielen Beispielen daf�r \autocite[vgl.][]{OS5/2011}    


\section{Motivation}

\section{Probleme}
