\chapter{Wiederverwendung von Software}

Bevor wir jedoch genauer auf die testgetriebene Wiederverwednung eingehen, m�chten wir zun�chst das Thema Wiederverwendung von Software allgemein betrachten. 

\section{Motivation}

Wie bereits eingangs erw�hnt stehen Softwarefirmen vor einem immer st�rkeren Konkurrenz-, Zeit- und Kostendruck. Die Anforderung bessere Software in k�rzerer Zeit zu geringeren Kosten zu erstellen kann nur mithilfe verschiedenster Methoden u.a. der gezielten Wiederverwendung von Software realisiert werden.

Durch Wiederverwendung bereits entwickelter Software kann die Entwicklungszeit eines neuen System signifikant gemindert werden. Neben der reinen Entwicklungszeit wird somit auch die Zeit f�r Test und evtl. Validierung gespart. Die Wiederverwendung erh�ht also die Produktivit�t. Die eingesparte Zeit wirkt sich nat�rlich auch positiv auf die Entwicklungskosten eines Projektes aus.

Des weiteren kann davon ausgegangen werden, das bereits exisitierende Softwarekomponenten eine h�here Zuverl�ssigkeit besitzen, da Fehler die evtl. im Entwurf oder der Implementierung gemacht wurden, durch den bereits stattgefundenen Einsatz aufgefallen und behoben sind. Dadurch wird implizit die Qualit�t der neu entwickelten Software erh�ht.    

Gerade die h�here Zuverl�ssigkeit und Qualit�t der Software, sowie die Einsparung von Zeit und Kosten f�hren zu einer besseren Planbarkeit f�r das gesamte Projekt.

Ein weiterer Vorteil ist, dass die Wartbarkeit von Software wesentlich erleichtert wird da evtl. auftretende Fehler in der wiederverwendeten Komponente nur einmal und nicht n-fach behoben werden m�ssen.

Die Vorteile der Wiederverwendung von Software sind also eindeutig und nicht von der Hand zu weisen. 



\section{Was bedeutet Wiederverwendung?}

Da bisher der Begriff der Wiederverwendung bisher recht allgemein und abstrakt benutzt wurde, werden wir den Begriff zun�chst definieren und die verschieden Ebenen der Wiederverwendung von Software erl�utern.

\subsection{Begriffsdefinition}
In der Literatur finden sich vielz�hlige und oft recht unterschiedliche Definitionen f�r den Begriff der Software-Wiederverwendung.

So beschreibt [krueger] den Begriff sehr allgemein wie folgt: 
\begin{description}
	\item[Software Reuse, \autocite{Kru92}] \enquote{ Software reuse is the process of creating software systems from existing software
rather than building software systems from scratch.} 
\end{description}

�hnlich beschreibt \autocite{diaz} den Begriff Software Reuse:
\begin{description}
	\item[Software Reuse, \autocite{diaz}] \enquote{ Software reuse is the use of existing software components to construct a new system.} 
\end{description}

Eine eher technische Definition findet sich in [] 

Wie obige Definitionen zeigen kann unter Wiederverwendung von Software weit mehr als die reine  Wiederverwendung von Code verstanden werden. Es kann zwischen folgenden Abstraktionsebenen unterschieden werden.
  
Die testgetriebene Wiederverwendung fokussiert die Wiederverwendung von Code-Fragmenten. Daher lehnen wir uns an die Definition von [] wenn im Folgenden der Begriff  Wiederverwendung von Software gebraucht wird.

\subsection{Abstraktionsebenen}

\subsection{Aspekte der Wiederverwendung}
   
   
\section{Probleme}

Die Wiederverwendung von Software ist seit beginn der Softwareentwicklung ein zentrales Thema. Der Begriff wurde erstmals von McIlroy 1968 gepr�gt. Er beschreibt in [] die damals katastrophale Lage der Softwareentwicklung und die Wiederverwendung als m�glichen Ausweg aus der Softwarekrise.
 
Seitdem erhoffte man sich mit neuen Paradigmen der Softwareentwicklung stets auch die Wiederverwendung von Softwarekomponenten zu erleichtern und zu verbessern. Objektorientierte Programmierung und service-orientierte Architekturen sind nur wenige von vielen Beispielen daf�r \autocite[vgl.][]{OS5/2011}

Und auch heutzutage gestaltet sich die Wiederverwendung von Software recht problematisch.[vgl] Doch warum l�sst sich die recht einfache Idee der Wiederverwendung mit ihren vielen Vorteilen nur so schwer in die Praxis umsetzen? 

Das haupts�chliche Problem l�sst sich recht einfach und anschaulich formulieren. 

\begin{center}
Software wiederverwendet hei�t Software wiederfinden.
\end{center}

Aufgrund der rasant gestiegenen Open Source entwicklung ist die Anzahl an m�glichen Kandidaten enorm gestiegen. War zu den Anf�ngen der Wiederverwendung von Software das Problem dass es zu wenige m�gliche Kandidaten gab, so hat sich dies durch die Open Source Entwicklung ins Gegenteil gewendet. Jedoch ist dieser Umstand genauso problematisch. Die enorme Gr��e von Software-Repos kann von einem Entwickler kaum noch Inhaltlich durchschaut werden kann. Allein das aktuelle JDK umfasst ca. 3800 Basisklassen. Zieht man weitere bekannte Frameworks in die Rechnung mit ein kommt allein im Java-Umfeld auf 10.000 Klassen. 

Selbst Firmen-intern gestaltet sich das Wiederfinden von Software Komponenten oft schwierig. Oftmals geht entsprechendes Wissen �ber die Existenz entsprechender Komponenten mit dem Ausscheiden von Mitarbeiten verloren, oder wird nach Beendigung des Projekt schlicht vergessen. 