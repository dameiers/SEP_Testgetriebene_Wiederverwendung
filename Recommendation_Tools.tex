\chapter{Recommendation Tools}
%ich bin nicht wirklich zufrieden mit der Position und der Text ist nicht wirklich besonders, kA warum die das eigentlich im Artikel drin haben. Nutzt nicht wirklich was.
In diesem Kapitel werden kurz Werkzeuge vorgestellt, die es erleichtern Code wiederzuverwenden. Sie benutzen keine testgetriebene Wiederverendung, allerdings können sie als Inspiration betrachtet werden um solche Werkzeuge zu entwickeln. Die Informationen wurden aus \autocite{OS5/2011} und \autocite{Hum08} entnommen.

\section{CodeBroker}
CodeBroker ist ein Werkzeug, das in Emacs integriert werden kann. Es arbeitet im Hintergrund und schlägt dem Benutzer Komponenten vor, die er womöglich gebrauchen kann. Das Werkzeug hat allerdings zwei große Nachteile. Zum einen sucht es die Empfehlungen mit Hilfe von Kommentaren. Somit muss der Benutzer ungeschriebenen Code kommentieren, was den Arbeitsaufwand erhöht. Zum anderen gibt es zu wenige Komponenten, die überhaupt vorgeschlagen werden können.

\section{Rascal}
Rascal ist ein Eclipse Plugin, welches dessen Auto-Complete Funktion erweitert. Demnach kann es dem Benutzer empfehlen, wann er welche Methoden wie benutzen soll. Um die Vorschläge zu erstellen wird Collaborative Filtering benutzt, diese Technik wird häufig in Onlineshops benutzt um deren Kunden Artikel vorzuschlagen.

\section{Prospector, ParseWeb, Strathcona}
Diese Werkzeuge arbeiten auf vorhandenem Code und extrahieren aus diesem genau die Fragmente, in denen die gewünschte Bibliothek zuvor schon einmal in einem ähnlichen Kontext angesprochen wurde.
%(!!ABGESCHRIEBEN. Wie soll das konkret aussehen?) 
