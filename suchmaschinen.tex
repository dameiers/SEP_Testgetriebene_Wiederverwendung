\chapter{Software suchen}

Die f�r Menschen nicht mehr zu handelnde Gr��e von Softwarerepositories f�hrt zu der Notwendigkeit das die Suche nach m�glichen Kandidaten maschinell erfolgen muss.

Dies ist jedoch nicht als Nachteil zu sehen, da durch die maschinelle Unterst�tzung es f�r den Entwickler wesentlich leichter wird nach Komponenten beliebiger Funktionalit�t zu suchen. 

Daher verwundert es nicht, dass die Idee einer maschinellen Suche nach Software nicht neu ist, sondern bereits seit �ber 25 Jahren exisitert. [vgl Quelle] Doch lange Zeit wurden keine nennenswerten Fortschritte in der Entwicklung von Suchmaschinen gemacht, da aufgrund der geringen verf�gbaren Menge an Software die bis dato entwicklten Systeme vollkommen ausreichend waren. Erst mit der Open Source Bewegung wurde deutlich, dass weiterer Forschungsbedarf besteht. Die immer gr��ere Menge an Software zeigte, dass bisherige Suchmaschinen zu schlechte Ergebnisse lieferten und f�r die Suche viel zu lange brauchten. [Quelle]. Auch heute noch besteht Forschungsbedarf um die Suchergebnisse qualitativ besser zu machen. 

Die oftmals schlechte Qualit�t von Suchergebnissen liegt vor allem darin begr�ndet, dass sich die Suche nach Software wesentlich von anderen Suchproblemen wie zb. der Internetsuche unterscheidet.

Es ist leicht einzusehen, dass eine rein textbasierte Suche wie sie in Internetsuchmaschinen zum Einsatz kommt bei der Suche nach Software nur m��ige Ergebnisse liefert. 
Dieses Ph�nomen wird in \autocite[]{OS5/2011} wie folgt beschrieben: 
\enquote{So mag zwar der Suchbegriff ``Matrix'' durchaus Quelltexte liefern, die der Berechnung von Matrizen-Rechen-operationen dienen.Es ist allerdings beispielsweise bei Weitem nicht offensichtlich, welche Funktionalit�t im Einzelnen unterst�tzt wird, oder ob nicht unter Umst�nden nur der Begriff ``Matrix'' zuf�llig im Quellcode oder der Beschreibung eines Suchergebnisses auftaucht}

Wie obiges Beispiel zeigt ist die Suche nach Software oder genauer gesagt das Formulieren einer guten Suchanfrage eine nicht triviale und komplexe Angelegneheit. Hinzukommt, dass eine zu genaue Suchanfrage das Suchergebnis zu stark einschr�nkt und somit unter Umst�nden evtl. passende oder vielversprechende Kandidaten nicht weiter ber�cksichtigt werden.

Im Folgenden werden verschiedene Suchtechniken vorgestellt, die die Suchm�glichkeiten der Softwaresuche verbessern k�nnen.

\section{Suchtechniken}

\subsection{Textbasierte Suche}

\subsection{Beschreibende Suchverfahren}

\subsection{Strukturbasierte Suchverfahren}

\subsection{Bedeutungsbasierte Suchverfahren}

\section{Beispiele}

