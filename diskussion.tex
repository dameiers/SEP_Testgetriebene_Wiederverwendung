\chapter{Diskussion}

Zuletzt m�chten wir in diesem Kapitel eine Nachbetrachtung und pers�nliche Meinung zum Thema testgetriebene Wiederverwendung und den vorgestellten Tools anstellen.

Die Ideen der testgetriebene Wiederverwendung stellen eine hervorragende M�glichkeit dar, die Wiederverwendung in den praktischen Alltag der Softwareentwicklung zu integrieren.
Das gezielte Ausnutzen der beim Test-First Prinzip anfallenden Testf�lle bevor �berhaupt eine Zeile Code entwickelt wurde, ist der Hauptvorteil an der testgetriebenen Wiederverwendung. Nur so l�sst sich die Qualit�t der m�glichen Wiederverwendungskandidaten ohne Mehraufwand erheblich verbessern. Hinzu kommt, dass das Test-First Prinzip sowieso einen festen Stellenplatz in der heutigen Softwareentwicklung gefunden hat, was auch daran belegt werden kann, dass nahezu alle modernen und gerade agile Entwicklungsmodelle dieses Prinzip verfolgen.
Ebenfalls wird erst durch das Prinzip der testgetriebenen Wiederverwendung, eine bessere Integration in Entwicklungsumgebungen sowie eine Automatisierung von Anfragen m�gich. Erfeulicherweise werden gerade diese Ans�tze von den vorgestellten Tools verfolgt. Dies ist unserer Meinung nach auch der richtige Weg.

Jedoch steht und f�llt die Idee der testgetriebenen Wiederverwendung auch mit Tools die diese erst m�glichen. Und hier scheint zumindest nach heutigem Stand das Problem zu liegen. S�mtliche vorgestellten Tools machen zur Zeit einen eher provisorischen Eindruck und sind noch weit von einem wirklichen allt�glichen Praxiseinsatz entfernt. Sei es die Dauer von Suchanfragen, oder auch die angesprochenen Probleme bzgl. der Qualit�t. Bei der Nutzung der Tools wird schnell klar, dass hier noch einiges an Forschungsbedarf besteht.

Neben diesen eher praktisch orientierten Problemen, hat uns aber noch ein weiteres Problem besch�ftigt. Welche Auswirkungen kann das automatisierte Absetzen von Suchanfragen und der einfachen M�glichkeiten komplette funktionale Module in das eigene Projekt zu integrieren, haben? Ein bereits erw�hntes Problem sind dabei die lizenrechtlichen Fragen. Dieses Problem ist unserer Meinung nach nur zu l�sen, falls individuell organisatorische Regelungen getroffen werden, um Lizenzverst��e und die damit verbundenen rechtlichen Konsequenzen zu vermeiden. Als weiteres Problem fiel uns auf, das gerade bei �ffentlichen Suchmaschinen wie Merobase, in denen jeder Softwaremodule bereitstellen kann, durchaus die M�glichkeit besteht, dass die Qualit�t beeinflusst wird. Schlimmer noch, die integrierten Module k�nnen unerw�snchte Seiteneffekte haben die es zu vermeiden gilt. Um dies zu gew�hrleisten m�ssten Pr�fungen f�r jedes integrierte Modul durchgef�hrt werden, was wiederum mit erheblichem Aufwand verbunden sein kann.

Als dies f�hrte uns zu dem Entschluss, dass besonders die in \autocite{OS5/2011} genannte Kombination mit einer �ffentlichen Suchmaschine eher f�r Forschungszwecke gedacht sind. Unserer Meinung nach werden Tools wie zb. CodeConjurer besonders interessant, um in gro�en und mittleren Unternehmen ein firmeninternes System zur Wiederverwendung bereit zu stellen. In diesem Fall m�sste statt auf die �ffentlichen Suchmaschine Merobase, auf firmeninterne Softwarerepositories zugegriffen werden.

Es bleibt abzuwarten ob die vorgestellten Tools weiter entwickelt werden und sich verbessern k�nnen. W�nschenswert w�re es jedenfalls.    
