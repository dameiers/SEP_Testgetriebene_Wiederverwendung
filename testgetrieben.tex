\chapter{Testgetriebene Wiederverwendung}

\section{Idee testgetriebene Wiederverwendung}

Um Komponenten wiederzuverwerten, muss der Benutzer diese �ber Suchmaschinen suchen, wobei die gefundenen Komponenten manuell �berpr�ft und getestet werden m�ssen, ehe diese in einem Projekt verwendet werden k�nnen. Die Suchmaschinen liefern allerdings viele Komponenten, die den Anforderungen nicht entsprechen, sogenannte "False Positives". \autocite[vgl.][10]{OS5/2011} Diese False Positives kommen unter anderem zustande, da die Menge an Open-Source Code beachtlich ist und dieser Code durchaus wiederverwendet werden kann. Allerdings arbeiten die Suchmaschinen nicht gut genug um s�mtliche falschen Ergebnisse herauszufiltern. Dies macht das manuelle heraus filtern sehr zeitaufwendig, so dass der Zeitgewinn bei der Wiederverwendung gering oder sogar negativ ausf�llt und es schneller w�re die entsprechende Komponente neu zu schreiben. Diese Ungewissheit ob der Zeit gewonnen wird oder nicht, ist ein Argument gegen die Wiederverwendung, da durchaus viel Zeit verschwendet werden kann.

Hier kommt nun die testgetriebene Wiederverwendung ins Spiel. Diese versucht die Anzahl der False Positives zu verringern, indem die gefundenen Komponenten automatisch gepr�ft und getestet werden. Dem Benutzer  werden nur die vielversprechenden Komponenten angezeigt, also jene die den Test �berstanden haben. Somit wird dem Benutzer Arbeit abgenommen und das Wiederverwenden, aus Sicht des Benutzers, ist nicht mehr so zeitaufw�ndig. Die Idee dass die Komponenten automatisch getestet werden, ist dank Unit-Tests und den agilen Entwicklungsmethoden nicht so abwegig. Im konkreten Fall bedeutet dies, dass der Benutzer einen Unit-Test f�r eine Komponente schreibt, die noch nicht entwickelt wurde. Dies ist eine Herangehensweise, die besonders bei agilen Entwicklungsmethoden h�ufig anzutreffen ist. Da der Test den Aufbau der gesuchten Komponente beschreibt, kann eine Suchmaschine diesen als Suchanfrage verwenden, um nach �hnlichen Komponenten zu suchen. Danach k�nnen die gefundenen Komponenten mit jenem Test getestet werden und der Benutzer erh�lt als Suchergebnis nur noch die Komponenten, die den Test erfolgreich bestanden haben. Die gefundenen Ergebnisse passen per Definition ins Projekt des Benutzers, da sie den von ihm erstellten Test bestanden haben. 


\section{Vergleich und Stand von Tools}
Die in \autocite[11]{OS5/2011} erw�hnten Werkzeuge, die eine testgetriebene Wiederverwendung erm�glichen, sind:
\begin{itemize}
	\item Code Genie
	\item S6
	\item Code Conjurer
\end{itemize}

Diese 3 Werkzeuge werden im Folgenden, vorgestellt wobei auf Code Genie und S6 nur kurz eingegangen wird, und Code Conjurer ausf�hrlicher betrachtet wird.

\subsection{Code Genie}
Die in \autocite[2]{Ota07} angegebene URL zu Code Genie existiert nicht mehr und es wurde auch sonst keine M�glichkeit gefunden um Code Genie herunterzuladen und zu testen. Laut \autocite[11]{OS5/2011} ist Code Genie ein Eclipse-Plug-In das die Suchmaschine Sourcerer benutzt. Allerdings verlangtes ein "manuelles Integrieren und/oder Testen der Kandidaten in einer Kopie des eigenen Eclipse-Projektes."

\subsection{S6}
S6 ist ein Web-Interface das �ber \autocite{s6search} erreichbar ist. Das Interface hat zwei gro�e Nachteile:
\begin{itemize}
	\item Es k�nnen nur einfache Testf�lle angegeben werden, die die Maske des Interfaces zul�sst. Kompliziertere Unit-Tests sind somit nicht realisierbar, da das Hochladen von Dateien nicht m�glich ist. 		Siehe in der Hilfe von: \autocite{s6search}
	\item Weiterhin fehlt eine Anbindung an eine Entwicklungsumgebung, wie etwa Eclipse, so dass das Arbeiten mit S6 recht umst�ndlich ist. \autocite[10]{s6search}
\end{itemize}
   


